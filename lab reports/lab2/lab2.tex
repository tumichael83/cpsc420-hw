\documentclass[a4paper, 12pt]{article}
\usepackage{amsmath}
\usepackage{amsfonts, amsthm, amssymb} % this has the proof environment
\usepackage[top=1in, bottom=1in, left=.8in, right=.8in]{geometry} % good margins

% images 
\usepackage{graphicx}
\graphicspath{ {./images/} }

%verilog code
\usepackage{xcolor}
\usepackage{listings}
\definecolor{vgreen}{RGB}{104,180,104}
\definecolor{vblue}{RGB}{49,49,255}
\definecolor{vorange}{RGB}{255,143,102}

\lstdefinestyle{verilog-style}
{
    language=Verilog,
    basicstyle=\small\ttfamily,
    keywordstyle=\color{vblue},
    identifierstyle=\color{black},
    commentstyle=\color{vgreen},
    numbers=left,
    numberstyle=\tiny\color{black},
    numbersep=10pt,
    tabsize=8,
    moredelim=*[s][\colorIndex]{[}{]},
    literate=*{:}{:}1
}

\makeatletter
\newcommand*\@lbracket{[}
\newcommand*\@rbracket{]}
\newcommand*\@colon{:}
\newcommand*\colorIndex{%
    \edef\@temp{\the\lst@token}%
    \ifx\@temp\@lbracket \color{black}%
    \else\ifx\@temp\@rbracket \color{black}%
    \else\ifx\@temp\@colon \color{black}%
    \else \color{vorange}%
    \fi\fi\fi
}
\makeatother

\usepackage{trace}

\newcommand{\paren}[1]{\left(#1\right )}
\newcommand{\set}[1]{\left \{#1\right \}}
\newcommand{\R}[0]{\mathbb R}
\newcommand{\Z}[0]{\mathbb Z}
\newcommand{\de}[0]{\text{\normalfont d}}
\newcommand{\mycomment}[1]{}


\begin{document}

\title{\vspace*{-.9in}Lab 2: Pipelined Muldiv Unit}
\date{Due: 2/22/2023}
\author{Michael Tu}
\maketitle
\hrule

\section{Abstract}
The purpose of this lab was to implement a pipelined, bypass enabled processor that can execute the PARC2 subset of x86-instructions. This provides enough functionally to run basic C code which does not require systemcalls. In particular, the processor also includes a pipelined multiplication/division unit to increase performance.

\section{Design Considerations:}
\textbf{Section 1:} For section 1, the main design change I needed was to add additional wires to capture additional branch conditions besides \texttt{bne}.
\\ \\
\textbf{Section 2:} For section 2, I modified my control unit to send bypass signals (\texttt{r[x]\_byp\_[X]hl}) to my datapath. The only cases in which one would need bypass values is when reading from the register file. Therefore, I added a mux input that combined the register files and bypass values. Finally, I added stalling for load-use dependencies, where a instruction must stall if it depends on a load.
\\ \\
\textbf{Section 3:} For section 3, I added two additional stages into the pipeline. Additionally, I needed to pass the \texttt{val} and \texttt{rdy} through the control units. The first additional stage ``Execute 2" does not perform any computation. Instead, it is only used to stall until the pipelined MulDiv unit completes. The second additional stage, ``Execute 3" is used to receive the output of the pipelined muldiv unit and pass it to the writeback stage.

\section{Testing}
In order to test, I tried repeatedly loading and reading from the same memory location. 

\section{Evaluation}
The benchmark results are below

\begin{center}
\begin{tabular}{ |p{3cm}|p{3cm}|p{3cm}|p{4.5cm}|  }
 \hline
 \multicolumn{4}{|c|}{pv2Stall Results} \\
 \hline
 Benchmark & Cycles & Instructions & IPC  \\
 \hline
 vvadd & 478 & 455 & 0.951883 \\
 cmplx-mult & 16718 & 1864 & 0.111497 \\
 masked-filter & 16174 & 4499 & 0.278162 \\
 bin-search & 3382 & 1279 & 0.378179 \\
 \hline
\end{tabular}
\end{center}

\begin{center}
\begin{tabular}{ |p{3cm}|p{3cm}|p{3cm}|p{4.5cm}|  }
 \hline
 \multicolumn{4}{|c|}{pv2Byp Results} \\
 \hline
 Benchmark & Cycles & Instructions & IPC  \\
 \hline
 vvadd & 473 & 455 & 0.961945 \\
 cmplx-mult & 15312 & 1864 & 0.121735 \\
 masked-filter & 13832 & 4499 & 0.325260 \\
 bin-search & 1749 & 1279 & 0.731275 \\
 \hline
\end{tabular}
\end{center}


\begin{center}
\begin{tabular}{ |p{3cm}|p{3cm}|p{3cm}|p{4.5cm}|  }
 \hline
 \multicolumn{4}{|c|}{pv2long Results} \\
 \hline
 Benchmark & Cycles & Instructions & IPC  \\
 \hline
 vvadd & 473 & 455 & 0.961945 \\
 cmplx-mult & 2662 & 1864 & 0.700225 \\
 masked-filter & 6058 & 4499 & 0.325260 \\
 bin-search & 1749 & 1279 & 0.731275 \\
 \hline
\end{tabular}
\end{center}
\section{Discussion}
From the benchmarks provided, we can see that bypassing provides a significant performance improvement over stalling, up to an almost 2$\times$ improvement. However, we see that even greater performance improvements can come by preventing the pipeline from stalling for long multiplications. This leads to a nearly order of magnitude improvement, especially for the complex multiplication test which involves many multiplications.

\section{Figures}
This is the datapath for part 2, the important difference is the addition of two muxes which allow for bypass values into the op0 and op1 muxes.


This is the datapath for part 2, the important difference is the addition of two muxes which allow for bypass values into the op0 and op1 muxes.


\end{document}